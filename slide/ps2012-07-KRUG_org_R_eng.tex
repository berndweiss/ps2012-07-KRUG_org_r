% Created 2012-07-09 Mo 10:31
\documentclass[bigger]{beamer}
\usepackage[utf8]{inputenc}
\usepackage[T1]{fontenc}
\usepackage{fixltx2e}
\usepackage{graphicx}
\usepackage{longtable}
\usepackage{float}
\usepackage{wrapfig}
\usepackage{soul}
\usepackage{textcomp}
\usepackage{marvosym}
\usepackage{wasysym}
\usepackage{latexsym}
\usepackage{amssymb}
\usepackage{hyperref}
\tolerance=1000
\usepackage{color}
\usepackage{listings}
\institute{\url{http://berndweiss.net} \\ \url{bernd.weiss@uni-koeln.de}}
\mode<beamer>{\usecolortheme{seagull}}
\usepackage{attrib}
\usepackage[T1]{fontenc}
\usepackage{tikz}
\usepackage[english]{babel}
\usepackage{graphicx}
\usepackage{tabularx}
\usepackage{amsmath}
\usepackage{bm}
\usepackage[babel]{csquotes}
\usepackage{blindtext} 
\usepackage{helvet} %\usepackage{lucidabr}
\AtBeginSection[]{ \begin{frame} \frametitle{Topic} \begin{footnotesize} \tableofcontents[currentsection, currentsubsection] \end{footnotesize} \end{frame} }
\AtBeginSubsection[]{ \begin{frame} \frametitle{Topic} \begin{footnotesize} \tableofcontents[currentsection, currentsubsection] \end{footnotesize} \end{frame} }
\hypersetup{colorlinks=true, urlcolor=cyan, linkcolor=black}
\providecommand{\alert}[1]{\textbf{#1}}

\title{A brief introduction to\newline reproducible research with\newline Emacs' Org mode and R}
\author{Bernd Weiß}
\date{\footnotesize{2012-07-06}}
\hypersetup{
  pdfkeywords={},
  pdfsubject={},
  pdfcreator={Emacs Org-mode version 7.8.11}}

\begin{document}

\maketitle

\begin{frame}
\frametitle{Outline}
\setcounter{tocdepth}{3}
\tableofcontents
\end{frame}


















\definecolor{dkgreen}{rgb}{0,0.5,0}
\definecolor{dkred}{rgb}{0.5,0,0}
\definecolor{gray}{rgb}{0.5,0.5,0.5}

\lstset{basicstyle=\ttfamily\bfseries\footnotesize,
morekeywords={virtualinvoke},
%%keywordstyle=\color{blue},
%%ndkeywordstyle=\color{red},
commentstyle=\color{dkred},
%%stringstyle=\color{dkgreen},
numbers=left,
numberstyle=\ttfamily\tiny\color{gray},
stepnumber=1,
numbersep=10pt,
backgroundcolor=\color{white},
tabsize=4,
showspaces=false,
showstringspaces=false,
xleftmargin=.23in
}




\section{Disclaimer and Acknowledgment}
\label{sec-1}
\subsection{}
\begin{frame}
\frametitle{Disclaimer and Acknowledgment}
\label{sec-1-1-1}

\begin{itemize}
\item I am by no means an expert user of Org mode or Org Babel. However, I know
  enough to acknowledge the incredible work of Carsten Dominik (created Org in 2003), Eric Schulte and Dan Davison (developed the Babel functionality within Org).
\item All materials can be found on github: \href{https://github.com/berndweiss/ps2012-07-KRUG_org_r}{https://github.com/berndweiss/ps2012-07-KRUG\_org\_r}
\end{itemize}
\end{frame}
\section{Introduction}
\label{sec-2}
\subsection{Motivating problem}
\label{sec-2-1}
\begin{frame}
\frametitle{Motivating problem}
\label{sec-2-1-1}


\begin{itemize}
\item Writing an empirical research paper
\begin{itemize}
\item is almost never a linear process.
\item also means that your colleagues need to trust you. However, how can I make them
    trust me?
\end{itemize}
\item So, we need a tool that provides
\begin{itemize}
\item \enquote{dynamic elements}, i.e. updates tables and graphs every time the data gets updated.
\item full access to code to check for correctness (i.e. \enquote{reproducible research}).
\end{itemize}
\end{itemize}
\end{frame}
\begin{frame}
\frametitle{Literate programming and reproducible research}
\label{sec-2-1-2}


\begin{description}
\item[\textbf{Literate programming (LP)}] Enhances traditional software development by
     embedding code in explanatory essays and encourages treating the
     act of development as one of communication with future
     maintainers.
\item[\textbf{Reproducible research (RR)}] Embeds executable code in research reports
     and publications, with the aim of allowing readers to re-run the
     analyses described.
\end{description}

     
(Source: Schulte et al. 2012: 3)
\end{frame}
\begin{frame}
\frametitle{Existing tools}
\label{sec-2-1-3}



\begin{center}
\begin{tabular}{l|ccccc}
                                       &           &       &  \LaTeX{}  &  HTML    &                                                               \\
 Tool                                  &  LP       &  RR   &  Export    &  Export  &  Language                                                     \\
\hline
 \hyperref[latex-proglang]{Javadoc}    &  partial  &  no   &  no        &  yes     &  \hyperref[latex-proglang]{Java}                              \\
 \hyperref[latex-proglang]{POD}        &  partial  &  no   &  no        &  yes     &  \hyperref[latex-proglang]{Perl}                              \\
 \hyperref[latex-proglang]{Haskell}    &  partial  &  no   &  yes       &  yes     &  \hyperref[latex-proglang]{Haskell}                           \\
 \hyperref[latex-proglang]{noweb}      &  yes      &  no   &  yes       &  yes     &  any                                                          \\
 \hyperref[latex-proglang]{cweb}       &  yes      &  no   &  yes       &  yes     &  \hyperref[latex-proglang]{C}/\hyperref[latex-proglang]{C++}  \\
 \hyperref[latex-proglang]{Sweave}     &  partial  &  yes  &  yes       &  yes     &  \hyperref[latex-proglang]{R}                                 \\
 \hyperref[latex-proglang]{SASweave}   &  partial  &  yes  &  yes       &  yes     &  \hyperref[latex-proglang]{R}/\hyperref[latex-proglang]{SAS}  \\
 \hyperref[latex-proglang]{Statweave}  &  partial  &  yes  &  yes       &  yes     &  any                                                          \\
 \hyperref[latex-proglang]{Scribble}   &  yes      &  yes  &  yes       &  yes     &  \hyperref[latex-proglang]{scheme}                            \\
 Knitr                                 &  partial  &  yes  &  yes       &  yes     &  R                                                            \\
 dexy                                  &  yes      &  yes  &  yes       &  yes     &  any                                                          \\
 \hyperref[latex-proglang]{Org-mode}   &  yes      &  yes  &  yes       &  yes     &  any                                                          \\
\end{tabular}
\end{center}



(Source: Schulte et al. 2012: 6, modified and updated)
\end{frame}
\subsection{Emacs}
\label{sec-2-2}
\begin{frame}
\frametitle{Emacs -- a brief overview}
\label{sec-2-2-1}

\begin{itemize}
\item Emacs is a text editor
\item It is highly extensible and customizable (using a Lisp dialect called Emacs Lisp)
\item Emacs stands for \enquote{Editor MACroS}
\item Development began in the 1970s, the current version is 24.1
\item I use Emacs mostly for writing R and \LaTeX-code
\item Emacs knows content-sensitive editing (major and minor) \emph{modes}, including syntax coloring, for
  a variety of file types (R, \LaTeX, C, C++, Python, plain text, HTML, \ldots{})
\end{itemize}

(Source: \href{http://www.gnu.org/software/emacs/}{http://www.gnu.org/software/emacs/}, \href{http://en.wikipedia.org/wiki/Emacs}{http://en.wikipedia.org/wiki/Emacs})
\end{frame}
\begin{frame}
\frametitle{Screen shot of an Emacs session}
\label{sec-2-2-2}


\includegraphics[width = 0.8\linewidth]{e:/projects/confer/ps2012-07-KRUG_org_r/fig/f_emacs_screenshot.png}
\end{frame}
\subsection{Org mode}
\label{sec-2-3}
\begin{frame}
\frametitle{Org mode}
\label{sec-2-3-1}

\begin{itemize}
\item Org is an Emacs (major) mode for notes, planning, and authoring
\item Org files are plain-text files
\item I use Org for
\begin{itemize}
\item writing technical documents (can be exported to PDF via pdfLaTeX or HTML) or
    creating slides
\item organizing my professional and private life (appointments, todo lists, \ldots{})
\item maintaining my personal \enquote{knowledge base}
\item \ldots{}
\end{itemize}
\item More infos on \href{http://www.orgmode.org}{http://www.orgmode.org} (tutorials, videos, \ldots{})
\end{itemize}
\end{frame}
\begin{frame}[fragile,shrink = 25]
\frametitle{An example Org file (source file)}
\label{sec-2-3-2}



\lstset{language=org}
\begin{lstlisting}
#+TITLE: Hello World!
#+AUTHOR: Martin Mosquito
#+DATE: <2012-07-06 Fr>

\thispagestyle{empty}

* Headline

  Lorem ipsum dolor sit amet, consetetur sadipscing elitr, sed 
  diam nonumy eirmod tempor invidunt ut labore et dolore magna 
  aliquyam erat, sed diam voluptua. At vero eos et accusam et 
  justo duo dolores et ea rebum. Stet clita kasd gubergren, no 
  sea takimata sanctus est Lorem ipsum dolor sit. 

  A list:
  - Item 1
  - Item 2  

** Subsection 

   Lorem ipsum dolor sit amet, consetetur sadipscing elitr, 
   sed diam nonumy eirmod tempor invidunt ut labore et dolore 
   magna aliquyam erat, sed diam
   
   \begin{equation}
   y = \alpha + \beta_1 x_1 + \epsilon
   \end{equation}
\end{lstlisting}
\end{frame}
\begin{frame}
\frametitle{An example Org file (exported to PDF)}
\label{sec-2-3-3}


\includegraphics[clip, width = 0.6\linewidth]{e:/projects/confer/ps2012-07-KRUG_org_r/fig/org_example.pdf}
\end{frame}
\subsection{Babel: Active code in Org mode}
\label{sec-2-4}
\begin{frame}
\frametitle{Babel: Active code in Org mode}
\label{sec-2-4-1}


\begin{quote}
Babel is about letting many different languages work together. Programming languages live in blocks inside natural language Org-mode documents. A piece of data may pass from a table to a Python code block, then maybe move on to an R code block, and finally end up embedded as a value in the middle of a paragraph or possibly pass through a gnuplot code block and end up as a plot embedded in the document.
\end{quote}

(Source: \href{http://orgmode.org/worg/org-contrib/babel/intro.html}{http://orgmode.org/worg/org-contrib/babel/intro.html})
\end{frame}
\begin{frame}[fragile]
\frametitle{Using code with(in) Org mode}
\label{sec-2-4-2}


  \usetikzlibrary{shapes,arrows,shadows,decorations,decorations.text,through}
  \tikzstyle{page} = [rectangle, draw, text width=9em,
  text centered, rounded corners,
  node distance=3cm, minimum height=1em,
  font=\tiny,
  fill=blue!20,
  general shadow={
    fill=black!30,
    shadow xshift=0.16cm,
    shadow yshift=-0.16cm
  },
  very thick,
  draw=blue]
  \begin{figure}[t!]
    \centering

%% Scale TIkZ figure
%% Source: http://stackoverflow.com/questions/2239328/control-font-size-in-graphics-in-latex-when-scaling-in-minipage-subfig
    \begin{tikzpicture}[->,>=stealth', shorten >=1pt, auto, transform canvas={scale=0.55}]
      \node [page] (org) at (0,0) {
        \begin{center}
          \normalsize{Org-mode}
        \end{center}
  \begin{verbatim}
    * Plain Text Markup
    - prose composition
    - code composition
    - data analysis

  ,  #+begin_src C :tangle run.c
      int main(){
        return 0;
      }
  ,  #+end_src

  ,  #+headers: :results graphics
  ,  #+begin_src R :file fig.pdf
      plot(data)
  ,  #+end_src

  \end{verbatim}
      };

      \node [page] (htm) at (8,1) {
        \begin{center}
          \normalsize{HTML}
        \end{center}
  \begin{verbatim}
    <h1>Plain Text Markup</h1>
    <ul>
    <li>prose composition</li>
    <li>code composition</li>
    <li>data analysis</li>
    </ul>
  \end{verbatim}
      };

      \node [page] (tex) at (9,-1) {
        \begin{center}
          \normalsize{\LaTeX{}}
        \end{center}
  \begin{verbatim}
    \Section{Plain Text Markup}
    \begin{itemize}
    \item prose composition
    \item code composition
    \item data analysis
    \end{itemize}
  \end{verbatim}
      };

      \node [page, text width=6.5em] (src) at (-8,0) {
        \begin{center}
          \normalsize{Source Code}
        \end{center}
  \begin{verbatim}
    int main(){
      return 0;
    }
  \end{verbatim}
      };

      \node [text width=8em] (code-out) at (4,-6) {Embedded data and
        source code in arbitrary\\ languages};

      \node [text width=8em] (code-out) at (-4,-6) {Raw output,
        tabular data, figures, etc.};

      \path (org) edge [loop below] node {\small{Code Evaluation}} (
      \path (org) edge node {\small{Export}} (5.25,0);
      \path (org) edge node [above, text width=4.25em] {\small{Code\\  Extraction}} (-5.875,0);
      \path (org) edge node [below] {\small{(Weave)}} (5.25,0);
      \path (org) edge node [below] {\small{(Tangle)}} (-5.875,0);
    \end{tikzpicture}
  \end{figure}
\vspace{3cm}
\vfill
Source: Schulte et al. (2012)
\end{frame}
\section{Babel and R}
\label{sec-3}
\subsection{Introduction}
\label{sec-3-1}
\begin{frame}[fragile]
\frametitle{The structure of Org code blocks}
\label{sec-3-1-1}


\begin{small}

\begin{verbatim}
#+NAME: <name>
#+BEGIN_SRC <language> <switches> <header arguments>
 <body>
#+END_SRC
\end{verbatim}
\end{small}
\end{frame}
\begin{frame}[fragile,t, shrink = 15]
\frametitle{R code blocks I (exports code and results)}
\label{sec-3-1-2}
\begin{block}{R code block}
\label{sec-3-1-2-1}


\begin{verbatim}
This is a short sentence. This is another sentence. 
And, finally, here comes some R code:

#+BEGIN_SRC R
set.seed(1)
x1 <- rnorm(100) 
mean(x1) 
x1.sd <- sd(x1) 
#+END_SRC

#+RESULTS:
: [1] 0.1088874
\end{verbatim}
\end{block}
\begin{block}{Evaluated and exported R code block}
\label{sec-3-1-2-2}


This is a short sentence. This is another sentence. And, finally, here comes some R code:

\lstset{language=R}
\begin{lstlisting}
set.seed(1)
x1 <- rnorm(100)
mean(x1)
x1.sd <- sd(x1)
\end{lstlisting}

\begin{verbatim}
 [1] 0.1088874
\end{verbatim}
\end{block}
\end{frame}
\begin{frame}[fragile,t, shrink = 15]
\frametitle{R code blocks II (exports only results)}
\label{sec-3-1-3}
\begin{block}{R code block}
\label{sec-3-1-3-1}


\begin{verbatim}
This is a short sentence. Hey, there is another 
sentence. And, finally, here comes some R code:

#+BEGIN_SRC R :results output :exports results
set.seed(1)
x1 <- rnorm(100) 
mean(x1) 
x1.sd <- sd(x1) 
#+END_SRC

#+RESULTS:
: [1] 0.1088874
\end{verbatim}
\end{block}
\begin{block}{Evaluated and exported R code block}
\label{sec-3-1-3-2}


This is a short sentence.  Hey, there is another sentence. And, finally, here comes some R code:



\begin{verbatim}
 [1] 0.1088874
\end{verbatim}
\end{block}
\end{frame}
\begin{frame}[fragile]
\frametitle{R inline code blocks (source)}
\label{sec-3-1-4}


\begin{footnotesize}

\begin{verbatim}
The mean of x is 
src_R[:exports results :results value raw]{round(mean(x1), 3)}
and the standard error is 
src_R[:exports results :results value raw]{round(x1.sd, 3)}. 
Remember that the vectors =x1= and =x1.sd= have been defined 
on the previous slide in a separate code block.
\end{verbatim}
\end{footnotesize}
\end{frame}
\begin{frame}
\frametitle{R inline code blocks (evaluated)}
\label{sec-3-1-5}


The mean of x is 0.109
and the standard error is 
0.898. Remember that the
vector \texttt{x1} has been defined on the previous slide in a separate code block.
\end{frame}
\begin{frame}[fragile]
\frametitle{Creating a base graphics plot (source)}
\label{sec-3-1-6}



\begin{verbatim}
 1:  #+headers: :exports both 
 2:  #+headers: :results graphics
 3:  #+headers: :file img.pdf
 4:  #+begin_src R 
 5:  x <- rnorm(100)
 6:  hist(x)
 7:  #+end_src 
 8:  
 9:  #+RESULTS:
10:  [[file:img.pdf]]
\end{verbatim}
\end{frame}
\begin{frame}[fragile]
\frametitle{Creating a base graphics plot (evaluated)}
\label{sec-3-1-7}



\lstset{language=R}
\begin{lstlisting}
x <- rnorm(100)
hist(x, main = "")
\end{lstlisting}


\includegraphics[width = 0.6\linewidth]{../fig/f_baseplot.pdf}
\end{frame}
\begin{frame}[fragile]
\frametitle{Creating a \texttt{ggplot2} plot (Org source)}
\label{sec-3-1-8}



\begin{verbatim}
 1:  #+headers: :exports results 
 2:  #+headers: :results graphics
 3:  #+headers: :file f_ggplot2_ex.pdf
 4:  #+begin_src R 
 5:  library(ggplot2)
 6:  x <- data.frame(var = rnorm(100))
 7:  ggplot(aes(x = var), data = x) 
 8:        + geom_histogram()
 9:  #+end_src 
10:  
11:  #+RESULTS:
12:  [[file:f_ggplot2_ex.pdf]]
\end{verbatim}
\end{frame}
\begin{frame}
\frametitle{Creating a \texttt{ggplot2} plot (evaluated)}
\label{sec-3-1-9}


 

\includegraphics[width = 0.7\linewidth]{../fig/f_ggplot2_example.pdf}
\end{frame}
\subsection{Tangling and weaving}
\label{sec-3-2}
\begin{frame}
\frametitle{Tangling and weaving code}
\label{sec-3-2-1}


\begin{itemize}
\item \emph{Weaving} refers to the exportation of a mixed code/prose document to a format
  that can be read by a human (HTML, \LaTeX, Ascii, \ldots{}).
\item \emph{Tangling} means to extract only the code to a separate file
\end{itemize}

(Source: Schulte et al. 2012: 12)
\end{frame}
\begin{frame}[fragile]
\frametitle{Example of code tangling}
\label{sec-3-2-2}



\begin{verbatim}
#+BEGIN_SRC R :tangle ../src/example_tangled.R 
## Comment: Assign value 1 to variable x
x <- 1
#+END_SRC
\end{verbatim}
\end{frame}
\begin{frame}[fragile]
\frametitle{Example (cont'd): Tangled R code}
\label{sec-3-2-3}


Here is the content of \texttt{../src/example\_tangled.R}
\begin{block}{\texttt{../src/example\_tangled.R}}
\label{sec-3-2-3-1}

\begin{footnotesize}

\begin{verbatim}

## [[file:e:/projects/confer/ps2012-07-KRUG_org_r/slide/ps2012-07-KRUG_org_R.org::*Example%20of%20code%20tangling][Example-of-code-tangling:2]]

## Comment: Assign value 1 to variable x
x <- 1

## Example-of-code-tangling:2 ends here
\end{verbatim}
\end{footnotesize}
\end{block}
\end{frame}
\subsection{A short example}
\label{sec-3-3}
\begin{frame}
\frametitle{A short example}
\label{sec-3-3-1}


See \texttt{pub/d\_example.org} and \texttt{pub/d\_example.pdf}.
\end{frame}
\subsection{A brief Org example}
\label{sec-3-4}
\begin{frame}
\frametitle{A brief Org example}
\label{sec-3-4-1}


See \texttt{pub/d\_example.org} and \texttt{pub/d\_example.pdf}.
\end{frame}
\section{R and other languages (Python)}
\label{sec-4}
\subsection{}
\begin{frame}[fragile,shrink = 10]
\frametitle{Chaining (source)}
\label{sec-4-1-1}




\lstset{language=org}
\begin{lstlisting}
Define variable =y= in a Python code block: 
#+name: chainexamp
#+begin_src python :results value :session *python*
y = 10**4
y
#+end_src

#+RESULTS: chainexamp
: 10000

Then pass variable =y= to an R code block:
#+BEGIN_SRC R :var x=chainexamp :session *R2*
print(x)
print(x*2)
#+END_SRC

#+RESULTS:
: [1] 10000
: [1] 20000
\end{lstlisting}
\end{frame}
\begin{frame}[fragile]
\frametitle{Chaining (evaluated)}
\label{sec-4-1-2}


Define variable \texttt{y} in a Python code block: 

\lstset{language=Python}
\begin{lstlisting}
y = 10**4
y
\end{lstlisting}

\begin{verbatim}
 10000
\end{verbatim}


Then pass variable \texttt{y} to an R code block:

\lstset{language=R}
\begin{lstlisting}
print(x)
print(x*2)
\end{lstlisting}

\begin{verbatim}
 [1] 10000
 [1] 20000
\end{verbatim}
\end{frame}
\section{References}
\label{sec-5}
\subsection{}
\begin{frame}
\frametitle{References}
\label{sec-5-1-1}


\begin{itemize}
\item Schulte, Eric, Dan Davison, Thomas Dye, und Carsten Dominik. 2012. A
  Multi-Language Computing Environment for Literate Programming and Reproducible
  Research. Journal of Statistical Software 46: 1–24.
\item \href{http://orgmode.org/worg/org-contrib/babel/how-to-use-Org-Babel-for-R.html}{http://orgmode.org/worg/org-contrib/babel/how-to-use-Org-Babel-for-R.html}
\end{itemize}
\end{frame}

\end{document}
